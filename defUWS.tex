\begin{frame}{G\'en\'eralisation des protocoles pr\'ec\'edents}
  \begin{figure}
    \begin{tikzpicture}
      \node (bot)  at (0,0) {$\bot$};
      \node (A) at (-2,-2) {Proposition $A$, $\SK_{A}$};
      \node (B) at (2,-2) {Proposition $B$, $\SK_{B}$};
      \node (C) at (-4, -4) {Proposition $C$, $\SK_{C}$};
      \path (bot) edge [->] (A);
      \path (bot) edge [->] (B);
      \path (A) edge [->] (C);
    \end{tikzpicture}
  \end{figure}
\end{frame}





\begin{frame}{D\'efinition de Universal Witness Signature}
  \begin{block}{D\'efinition}
    Une Universal Witness Signature par rapport \`a un relation pre-order $Ord$ (qui poss\`ede un plus grand element $*$) est une paire d'algorithmes $(Setup, Delegate)$ qui v\'erifie les sp\'ecifications suivantes :
    \begin{itemize}
    \item $Setup(1^\lambda) \to (PP,SK_*)$: G\'en\'erer un param\`etre publique $PP$ et une clef de signature $SK_*$.
    \item $Delegate(PP, SK_A, A, B, w) \to SK_B$: Si $SK_A$ est une clef de signature valide de $A$ et $w$ est un t\'emoin de $B\leq A$, alors l'algorithme retournera $SK_B$, sinon il retournera $\bot$.
    \end{itemize}
  \end{block}
\end{frame}

%\begin{frame}{Remarques}
 %
 % \begin{tabular}{|c|c|c|}
 %   \hline
 %   & Sch\'ema classique de  & Sch\'ema UWS\\
 %   & signature hi\'erarchique & \\
 %   \hline
 %   Identit\'e  & $Id = (id_1, \dots, id_k)$ & $Id = id_k$\\
 %   \hline
 %   Clef de signature &  $SK_{ID}$ & $SK_{ID}$\\
 %   \hline
 %   Message  & $M = (id_1, \dots, id_k, m)$ & $M = m$ \\
 %   \hline
 %   Signature & $\sigma_m$ & $SK_m$\\
 %   \hline
 % \end{tabular}
 %
 % \begin{block}{}
 %   Algorithme de v\'erification :
 %   $$Verify(PP, m, \sigma_m)  =  Delegate(PP,SK_m,m,m, ``\forall x. x\leq x")$$
 % \end{block}
%\end{frame}



\begin{frame}{Contributions}
  \begin{enumerate}
  \item G\'en\'eralisation de plusieurs notions de sch\'ema de signatures.
  \item Premi\`ere construction de signature propositionnelle.
  \end{enumerate}

\end{frame}