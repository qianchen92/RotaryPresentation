\begin{frame}{Signature redactable: Signer un message partiel}
  \begin{figure}[H]

    \begin{tikzpicture}
      
      \node (TopSecret) at (0,0){\includegraphics[width = 0.3\textwidth]{./images/Top-Secret.jpg}};
      \node (File) [text width = 3cm,draw,thick,minimum width=1.5cm,minimum height=0.75cm] at (0,0) {32 January 2000, Darth Vader had a dinner with master Yoda.};
      \node (SignatureInit) [text width = 3cm,draw,thick,minimum width=1.5cm,minimum height=0.75cm]at (0,-2) {Signature valide du fichier initial};
      \node (File2) [text width = 3cm,draw,thick,minimum width=1.5cm,minimum height=0.75cm] at (6,0) {32 January 2000, Darth Vader had a dinner with \#\#\#\#\#\#\#\#\#.};
      \node (SignatureFinal)[text width = 3cm,draw,thick,minimum width=1.5cm,minimum height=0.75cm] at (6,-2) {Signature valide du nouveau fichier};
      \path (SignatureInit) edge[->]node[above]{sans $msk$} (SignatureFinal);
      
    \end{tikzpicture}

  \end{figure}

\end{frame}


\begin{frame}{Signature propositionnelle: Signer logiquement}
  \begin{itemize}
  \item \'Element \`a signer: Formules du calcul propositionnel.
  \item Propri\'et\'e sp\'eciale de signature propositionnelle : $Delegate(PP, Sig(Prop_A),Prop_A,Prop_B,w_{Prop_A \Rightarrow Prop_B}) = Sig(Prop_B)$
  \end{itemize}

  Example:

  $(``A", \sigma_{A}) \leadsto (``A\wedge B", \sigma_{A\wedge B})$ 
\end{frame}
